\let\clearpage\relax
{\chapter{Domainanalyse}
\label{sec:domainanalyse}}
\section{Beschreibung der Domäne}
Die Domäne umfasst klinische Dokumentensysteme, insbesondere die Erfassung, Speicherung, Verarbeitung und Bereitstellung medizinischer Daten innerhalb eines Krankenhauses. Dokumente liegen aktuell in unterschiedlichsten Formaten vor – strukturiert (z.B. HL7-Nachrichten), semi-strukturiert (PDFs, CDA-Dokumente) und unstrukturiert (gescannte Dokumente). Das Zusammenspiel verschiedener Systeme ist mangelhaft, häufig existieren redundante Prozesse und Medienbrüche. Die rechtlichen Rahmenbedingungen sind durch Datenschutzgesetze, Medizinprodukterecht und sektorspezifische IT-Standards geprägt.
\section{Ist-Situation}
Am betrachteten Universitätsklinikum bestehen diverse Insellösungen, die nicht vollständig integriert sind. Informationen sind auf mehrere Systeme verteilt, darunter das KIS, separate Labor- und Entlassungssysteme sowie teilweise analoge Archive. Eine Suche erfordert oft über zehn Klicks und mehrere Logins. Die Informationsaufbereitung ist unzureichend, was zu Verzögerungen in der Patientenversorgung führt. Dokumente im PDF-Format sind nicht durchsuchbar, eine semantische Verknüpfung zwischen verschiedenen Datenquellen existiert nicht.
\section{Zielraum}
Das geplante System soll bestehende Informationssysteme konsolidieren, Datenströme harmonisieren und eine zentrale, durchsuchbare Dokumentenbasis schaffen. Mithilfe von Methoden der natürlichen Sprachverarbeitung und Transfer Learning sollen relevante Entitäten erkannt und in nutzerspezifischen Dashboards visualisiert werden. Die Interaktion mit dem System soll multimodal erfolgen können – über Tastatur, Touch und Sprache. Eine vollständig inhouse betriebene Infrastruktur garantiert die Einhaltung datenschutzrechtlicher Anforderungen. Zudem soll das System HL7- und FHIR-kompatibel sein, um eine künftige Anbindung externer Komponenten zu ermöglichen. Die Domäne unterliegt spezifischen Regularien, darunter die Datenschutz-Grundverordnung (DSGVO), §75c SGB V (IT-Sicherheit im Krankenhaus), die Richtlinie MDR zur Zulassung medizinischer Softwareprodukte sowie der Technische Standard TR-03161 des Bundesamts für Sicherheit in der Informationstechnik (BSI). Technisch kommen etablierte Schnittstellen wie HL7, FHIR und CDA zur Anwendung. Die Einhaltung dieser Normen ist grundlegend für die Zulassung und den sicheren Betrieb des Systems.
