{\let\clearpage\relax
\chapter{Technologieevaluation}}
\label{sec:technologieevaluation}
6 Technologieevaluation und Konkurrenzanalyse
\section{Ziel und Vorgehen}
Ziel dieses Arbeitspakets war es, den aktuellen Stand der Technologieentwicklung im Bereich medizinischer Dokumentenverarbeitung zu analysieren, geeignete technische Komponenten für die geplante Entity-Recognition- und Retrieval-Plattform auszuwählen und die Lösung strategisch gegenüber bestehenden Markt- und Speziallösungen zu positionieren.\\
\underline{Das Vorgehen erfolgte in vier Schritten:}
\begin{enumerate}
	\item \textbf{Anforderungsdefinition:} Ableitung funktionaler und nicht-funktionaler Anforderungen aus Stakeholderinterviews und den Projektzielen (z.B. HL7/FHIR-Unterstützung, kontextsensitive Suche, Barrierefreiheit, Datenschutz)
	\item \textbf{Marktanalyse (State of the Art):} Untersuchung bestehender Systeme in den Bereichen Datenvirtualisierung, Business Intelligence (BI) und medizinische Visualisierung
	\item \textbf{Technologieevaluation:} Bewertung geeigneter Technologien mittels Scoring-Modell nach Funktionsumfang, Standardkonformität, Integrationsfähigkeit, Stabilität, Community-Support und Eignung für den klinischen Kontext
	\item \textbf{Konkurrenzanalyse:} Vergleich mit führenden Marktakteuren anhand eines Kiviat-Diagramms und einer SWOT-Analyse zur strategischen Positionierung
\end{enumerate}
