{\let\clearpage\relax
\chapter{Konkurrenzanalyse}}
\label{sec:konkurrenzanalyse}

\section{Hintergrund}
Die Konkurrenzanalyse wurde durchgeführt, um die Positionierung des geplanten Entity-Recognition- und Retrieval-Systems im Vergleich zu bestehenden Markt- und Speziallösungen zu bewerten. Ziel war es, funktionale und nicht-funktionale Schwerpunkte der Konkurrenz zu verstehen, um daraus Rückschlüsse für die eigene Systemarchitektur und Priorisierung im MVP (Minimum Viable Product) zu ziehen.
\section{Ergebnisse}
\begin{itemize}
	\item \textbf{Epic Systems}: Umfassendes KIS mit hoher Funktionsdichte, sehr guter Interoperabilität und etablierten Standards (HL7, FHIR); marktführend, jedoch extrem teuer, schwer einzuführen und wenig flexibel für spezifische Anpassungen.
	\item \textbf{Dedalus ORBIS}: Im deutschsprachigen Raum stark etabliert, bietet jedoch veraltete UI-Elemente und komplexe Workflows; Anpassung an klinikspezifische Prozesse ist aufwendig.
	\item \textbf{Microsoft Cloud for Healthcare}: Überzeugt mit moderner Technologie, sehr guter Interoperabilität und schneller Verfügbarkeit; hohe laufende Kosten und Cloud-Abhängigkeit machen es für rein interne Infrastrukturen weniger geeignet.
	\item \textbf{Averbis Health Discovery}: Spezialisiert auf Textanalyse und semantische Suche, bietet hohe Integrationsmöglichkeiten (HL7, FHIR) und moderate Kosten, ist aber funktional weniger umfassend.
\end{itemize}
Das geplante MVP liegt im Funktionsumfang deutlich unter den genannten Systemen, setzt aber auf eine gezielte Fokussierung: schnelle, fehlertolerante Suche, einfache Bedienbarkeit, rollenbasierte Dashboards und HL7-konforme Integration.
\section{Ableitung für das Projekt}
Die Konkurrenzanalyse verdeutlicht, dass die Entwicklung einer vollumfänglichen KIS-Alternative weder notwendig noch realistisch ist. Der Schwerpunkt des geplanten MVP liegt daher bewusst auf Fokussierung und Modularität. Erweiterte BI-Analysen oder vollautomatisierte KI-Prozesse werden in der ersten Projektphase nicht umgesetzt, um Implementierungsrisiken und Kosten zu reduzieren. Die Kombination aus HL7- und FHIR-Kompatibilität, einer schlanken, leicht bedienbaren Architektur und der Konzentration auf dokumentenzentrierte Informationsabrufe stellt ein Alleinstellungsmerkmal dar, das sich deutlich von den komplexen und schwerfälligen Gesamtlösungen der Konkurrenz abhebt.
Für die kommenden Projektphasen ist es entscheidend, bereits während der Pilotierung ein strukturiertes Feedbacksystem und gezielte Schulungskonzepte zu etablieren, um eine hohe Nutzerakzeptanz zu gewährleisten. Langfristig eröffnet der modulare Aufbau des Systems die Möglichkeit, es schrittweise an die Funktionsvielfalt der Konkurrenz heranzuführen, beispielsweise durch die Integration von KI-basierten Analysen oder mobiler Nutzung. Damit liefert die Konkurrenzanalyse nicht nur eine strategische Positionierung des Systems, sondern auch konkrete Anhaltspunkte für die Priorisierung im weiteren Projektplan: Im Mittelpunkt steht ein stabiles, nutzerzentriertes MVP, das später gezielt erweitert werden kann.
