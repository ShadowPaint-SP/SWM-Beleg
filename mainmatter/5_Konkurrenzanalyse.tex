{\let\clearpage\relax
\chapter{Konkurrenzanalyse}}
\label{sec:konkurrenzanalyse}

\section{Hintergrund}
Die Konkurrenzanalyse wurde durchgeführt, um die Positionierung des geplanten Entity-Recognition- und Retrieval-Systems im Vergleich zu bestehenden Markt- und Speziallösungen zu bewerten. Ziel war es, funktionale und nicht-funktionale Schwerpunkte der Konkurrenz zu verstehen, um daraus Rückschlüsse für die eigene Systemarchitektur und Priorisierung im MVP (Minimum Viable Product) zu ziehen.
\section{Ergebnisse}
\begin{itemize}
	\item \textbf{Epic Systems}: Umfassendes KIS mit hoher Funktionsdichte, sehr guter Interoperabilität und etablierten Standards (HL7, FHIR); marktführend, jedoch extrem teuer, schwer einzuführen und wenig flexibel für spezifische Anpassungen.
	\item \textbf{Dedalus ORBIS}: Im deutschsprachigen Raum stark etabliert, bietet jedoch veraltete UI-Elemente und komplexe Workflows; Anpassung an klinikspezifische Prozesse ist aufwendig.
	\item \textbf{Microsoft Cloud for Healthcare}: Überzeugt mit moderner Technologie, sehr guter Interoperabilität und schneller Verfügbarkeit; hohe laufende Kosten und Cloud-Abhängigkeit machen es für rein interne Infrastrukturen weniger geeignet.
	\item \textbf{Averbis Health Discovery}: Spezialisiert auf Textanalyse und semantische Suche, bietet hohe Integrationsmöglichkeiten (HL7, FHIR) und moderate Kosten, ist aber funktional weniger umfassend.
\end{itemize}
Das geplante MVP liegt im Funktionsumfang deutlich unter den genannten Systemen, setzt aber auf eine gezielte Fokussierung: schnelle, fehlertolerante Suche, einfache Bedienbarkeit, rollenbasierte Dashboards und HL7-konforme Integration.
\section{Ableitung für das Projekt}
Die Konkurrenzanalyse verdeutlicht, dass die Entwicklung einer vollumfänglichen KIS-Alternative weder notwendig noch realistisch ist. Der Schwerpunkt des geplanten MVP liegt daher bewusst auf Fokussierung und Modularität. Erweiterte BI-Analysen oder vollautomatisierte KI-Prozesse werden in der ersten Projektphase nicht umgesetzt, um Implementierungsrisiken und Kosten zu reduzieren. Die Kombination aus HL7- und FHIR-Kompatibilität, einer schlanken, leicht bedienbaren Architektur und der Konzentration auf dokumentenzentrierte Informationsabrufe stellt ein Alleinstellungsmerkmal dar, das sich deutlich von den komplexen und schwerfälligen Gesamtlösungen der Konkurrenz abhebt.
Für die kommenden Projektphasen ist es entscheidend, bereits während der Pilotierung ein strukturiertes Feedbacksystem und gezielte Schulungskonzepte zu etablieren, um eine hohe Nutzerakzeptanz zu gewährleisten. Langfristig eröffnet der modulare Aufbau des Systems die Möglichkeit, es schrittweise an die Funktionsvielfalt der Konkurrenz heranzuführen, beispielsweise durch die Integration von KI-basierten Analysen oder mobiler Nutzung. Damit liefert die Konkurrenzanalyse nicht nur eine strategische Positionierung des Systems, sondern auch konkrete Anhaltspunkte für die Priorisierung im weiteren Projektplan: Im Mittelpunkt steht ein stabiles, nutzerzentriertes MVP, das später gezielt erweitert werden kann.
\section{Evaluation der Teillösungen}
\subsection{Datenvirtualisierung}
Die Datenvirtualisierung stellt die Grundlage für die Integration der stark segmentierten Systemlandschaft dar. Vier Lösungen wurden analysiert:
Dre\begin{itemize}
	\item mio bietet SQL-basierte Abfragen und zahlreiche Konnektoren, richtet sich jedoch primär an Geschäfts- und Verwaltungsdaten und ist nicht auf den Gesundheitsbereich spezialisiert.
	\item CData Driver Technologies überzeugt durch die größte Zahl an Integrationen und eine FHIR-Anbindung, bleibt jedoch generisch ohne spezifische Sicherheitsfeatures für klinische Daten.
	\item TIBCO Data Virtualization, als „Leader“ im GigaOm Radar ausgezeichnet, unterstützt zahlreiche Standards, einschließlich HL7 und FHIR, ist jedoch nicht primär auf den Gesundheitssektor ausgerichtet.
	\item Infor Cloverleaf wurde speziell für das Gesundheitswesen entwickelt, unterstützt Standards wie DICOM, CDA, X12 und FHIR und erfüllt branchenspezifische Sicherheitsanforderungen (HIPAA, ISO 27001). Zudem ist Cloverleaf für die Kommunikation zwischen verschiedenen Gesundheitseinrichtungen ausgelegt.
\end{itemize}
Mit \textbf{46 von 48 Punkten} erhielt Infor Cloverleaf die höchste Bewertung und wurde als bevorzugte Lösung empfohlen.

\subsection{Business-Intelligence-Systeme}
Für die kontextbasierte Analyse und Visualisierung der Daten wurden BI-Tools bewertet:
\begin{itemize}
	\item \textbf{Microsoft Power BI} und \textbf{Tableau} sind etablierte Lösungen mit großem Funktionsumfang und einer breiten Community. Power BI bietet eine tiefe Integration mit Microsoft-Produkten, während Tableau durch flexible Dashboards und zahlreiche Healthcare-Implementierungen überzeugt.
	\item \textbf{Qlik Sense Healthcare} punktet durch eine native FHIR-Schnittstelle und wurde speziell für den Gesundheitsbereich angepasst.
	\item \textbf{Grafana}, eine Open-Source-Lösung mit vielen Plugins, eignet sich besonders für Zeitreihenanalysen, ist aber weniger auf komplexe klinische Analysen ausgelegt.
\end{itemize}Empfohlen werden \textbf{Tableau} oder \textbf{Qlik Sense}, beide mit \textbf{20 von 22 Punkten}, da sie sich sowohl durch Funktionsvielfalt als auch durch klinische Anwendungsbeispiele auszeichnen.

\subsection{Medizinische Datenvisualisierung}
Zur Darstellung und Analyse medizinischer Bilddaten wurden folgende Lösungen untersucht:
Philips HealthSuite Imaging ist eine hochverfügbare Unternehmenslösung mit 99,99\% Uptime, jedoch stark an AWS-Cloud-Infrastrukturen gebunden.
3D Slicer ist eine Open-Source-Plattform mit breiter Community, vielen Erweiterungen und Unterstützung für Windows, Mac und Linux.
MITK (Medical Imaging Interaction Toolkit) bietet ebenfalls Open-Source-Funktionalität, hat jedoch eine kleinere Community und weniger praxisnahe Erweiterungen.
3D Slicer erhielt 17 von 29 Punkten und wird aufgrund seiner Community und Erweiterbarkeit empfohlen, obwohl es in Punkto Integration in klinische Abläufe noch optimiert werden muss.

\section{Zusammenfassende Empfehlung}
Auf Grundlage der Bewertung ergibt sich folgendes bevorzugtes Technologiebündel:
\begin{itemize}
	\item \textbf{Datenvirtualisierung:} Infor Cloverleaf – aufgrund der hohen Punktzahl, Spezialisierung auf den Gesundheitsbereich und umfassender Standardunterstützung.
	\item \textbf{Business Intelligence:} Tableau oder Qlik Sense – dank ihrer leistungsfähigen Dashboard-Funktionen und klinischer Anpassbarkeit.
	\item \textbf{Medizinische Visualisierung:} 3D Slicer – wegen der offenen Architektur und der großen Entwicklercommunity.

\end{itemize}Die Kompatibilität dieser Lösungen mit der bestehenden Infrastruktur muss jedoch vor der finalen Implementierung überprüft werden, um potenzielle Integrationsrisiken zu vermeiden.
