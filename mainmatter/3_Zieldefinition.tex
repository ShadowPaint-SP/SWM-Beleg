{\let\clearpage\relax
\chapter{Zieldefinition}}
\label{sec:zieldefinition}
Die in der Domainanalyse erkannten Schwachstellen, werden nun in konkrete Projektziele überführt.
\section{Projektziel}

Ziel des Projekts ist die Entwicklung einer integrierten Intranet-Plattform zur intelligenten Suche, Kontextanalyse und Visualisierung medizinischer Dokumente und Patienteninformationen. Das geplante System soll verschiedene Subsysteme (KIS, Labor, Medikation, Entlassungsberichte, PDF-Dokumente) semantisch zusammenführen und durch KI-basierte Methoden wie Natural Language Processing (NLP), Transfer Learning und LLMs klinische Entitäten automatisch extrahieren und strukturieren.
Ein personalisiertes Dashboard, eine kontextbasierte Darstellung sowie ein KI-gestützter Assistent sollen den Arbeitsalltag des medizinischen Personals signifikant erleichtern, die Qualität der Datennutzung erhöhen und zur Digitalisierung der klinischen Dokumentation beitragen – unter Einhaltung höchster Datenschutz- und Sicherheitsstandards.
\section{Geschäftsziele}

\begin{center}
\begin{tabular}{|p{5cm}|p{10cm}|}
	\hline
	\textbf{Ziel} & \textbf{Beschreibung} \\
	\hline
	Digitalisierung und Reduktion von Papieraufwand & Nachhaltige Digitalisierung der Dokumentationsprozesse zur Reduktion physischer Archivflächen und Druckkosten. \\
	\hline
	Zukunftssicherheit & Technische und funktionale Skalierbarkeit hinsichtlich Nutzerzahlen, Endgeräte und neuer Datenquellen. \\
	\hline
	Sehr hohe Verfügbarkeit und Ausfallsicherheit & Sicherstellung der Verfügbarkeit auch im Notfall durch redundante Infrastruktur und Ausfallsicherung. \\
	\hline
	Benutzerfreundlichkeit & Klare, konsistente und barrierefreie Oberfläche zur Reduktion der kognitiven Last und Erhöhung der Akzeptanz. \\
	\hline
	Zuverlässige und kontextgerechte Darstellung & Automatisierte Selektion und Visualisierung relevanter Informationen pro Fall. \\
	\hline
	Budgetrahmen und Zeitplan & Einhaltung des Budgets von maximal 500.000€ bei einer Projektlaufzeit von drei Jahren, medienwirksamer Prototyp nach 12–18 Monaten. \\
	\hline
	Gesetzliche Konformität & Einhaltung gesetzlicher Aufbewahrungsfristen (10 Jahre), MDR, DSGVO, BSI TR-03161 und Barrierefreiheitsstandards. \\
	\hline
\end{tabular}
\end{center}



\section{Operative Ziele}
\begin{center}
\begin{tabular}{|p{5cm}|p{10cm}|}
	\hline
	\textbf{Ziel} & \textbf{Beschreibung} \\
	\hline
	Pilotprojekt auf einer Station & Durchführung eines Pilotprojekts mit 50 Nutzern und 100 Patienten zur Evaluation im Echtbetrieb. \\
	\hline
	Antwortzeiten unter 3 Sekunden & Schnelle Entscheidungsprozesse durch kurze Antwortzeiten bei Standardabfragen. \\
	\hline
	Weniger als 10 Klicks pro Informationsabruf & Deutliche Reduktion der Interaktionslast im Vergleich zum Ist-Zustand. \\
	\hline
	Zentraler Login mit Langzeit-Authentifizierung & Single Sign-On (SSO) mit Zwei-Faktor-Authentifizierung zur Erhöhung von Sicherheit und Benutzerfreundlichkeit. \\
	\hline
	Interne Datenverarbeitung & Ausschließliche Verarbeitung im gesicherten Intranet, kein Cloudzugriff oder Datenexport. \\
	\hline
	Kompatibilität mit Bestandssystemen & Ergänzung bestehender KIS-Komponenten ohne deren vollständige Ablösung. \\
	\hline
\end{tabular}
\end{center}


\section{Stakeholderziele}
\begin{center}
\begin{tabular}{|p{4cm}|p{5cm}|p{6cm}|}
	\hline
	\textbf{Stakeholder} & \textbf{Ziel} & \textbf{Beschreibung} \\
	\hline
	Medizinisches Personal & Zeitersparnis und Effizienz & Reduzierte Klicks, zentrale Suche und kontextbezogene Darstellung ermöglichen eine spürbare Zeitersparnis. \\
	
	& Intuitive Bedienung & Bedienbarkeit ohne umfassende Schulungen, anpassbare Dashboards entsprechend individueller Arbeitsstile. \\
	
	& Sprachsteuerung & Sprachbasierte Interaktion für hygienisch sensible Situationen, z.B. bei Visiten oder Operationen. \\
	
	& KI-gestützte Unterstützung & Ad-hoc-Suche in natürlicher Sprache, semantische Abfragen und Anomalieerkennung zur Unterstützung medizinischer Entscheidungen. \\
	\hline
	Klinikdirektion & Hohe Verfügbarkeit und Sicherheit & Ausfallsicherheit, Rechtemanagement mit Zwei-Faktor-Authentifizierung und differenzierter Rollensteuerung. \\
	
	 & Backup und Kostenkontrolle & Nutzung der vorhandenen Serverarchitektur für Backup und Recovery, Kostenkontrolle durch Meilensteine und Evaluationsphasen. \\
	
	 & Langfristige Skalierbarkeit & Sicherstellung von Skalierbarkeit und Wartbarkeit für den Klinikbetrieb. \\
	\hline
\end{tabular}
\end{center}


\section{Ergebnisziele}
\begin{center}
\begin{tabular}{|p{5cm}|p{10cm}|}
	\hline
	\textbf{Ergebnisziel} & \textbf{Messkriterium} \\
	\hline
	Antwortzeiten & Standardabfragen unter 3 Sekunden \\
	\hline
	Interaktionen pro Suche & Weniger als 10 Interaktionen \\
	\hline
	Zeitersparnis & Mindestens 30 \% im Klinikalltag \\
	\hline
	Nutzerzufriedenheit & Mindestens 75 \% positives Feedback im Pilottest \\
	\hline
	Prototyp und Pilot & Funktionaler Prototyp nach 6 Monaten, Pilotbetrieb nach spätestens 12 Monaten \\
	\hline
\end{tabular}
\end{center}


\section{Nicht-Ziele}
\begin{center}
	\begin{tabular}{|p{5cm}|p{10cm}|}
		\hline
		\textbf{Was nicht erreicht werden soll} & \textbf{Begründung} \\
		\hline
		Vollständiger Ersatz des KIS & Das bestehende KIS bleibt Hauptsystem für Patientenstammdaten und Behandlungsdokumentation. \\
		\hline
		Mobile Speicherung personenbezogener Daten & Schutz vor Datenverlust bei Geräteverlust. \\
		\hline
		Integration externer Mandanten im Pilotbetrieb & Fokussierung auf interne Pilotstation. \\
		\hline
		Komplette Neugestaltung der UI & Wahrung der Vertrautheit und Akzeptanz der Nutzer. \\
		\hline
		Automatisierte Entscheidungsfindung & Das System unterstützt, ersetzt aber keine klinische Verantwortung. \\
		\hline
		Autorisierte Löschung kritischer Daten & Vermeidung von Sicherheits- und Compliance-Risiken. \\
		\hline
	\end{tabular}
\end{center}