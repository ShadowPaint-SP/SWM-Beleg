\setcounter{chapter}{0} % sets the number of this chapter to 0

\chapter{Einleitung}
\label{sec:einleitung}
Die zunehmende Digitalisierung im Gesundheitswesen eröffnet vielfältige Möglichkeiten zur Effizienzsteigerung, besseren Informationsbereitstellung und verbesserten Patientenversorgung. In medizinischen Einrichtungen ist jedoch häufig eine heterogene IT-Landschaft vorzufinden, in der Dokumente aus verschiedenen Quellen unstrukturiert abgelegt werden. Daraus resultieren Probleme bei der Informationssuche, Prozessineffizienzen sowie eine eingeschränkte Nachnutzung der Dokumentation. 
Ziel dieses Projekts ist die Planung eines Entity-Recognition- und Retrieval-Systems (ER-System) zur Strukturierung und Auffindung medizinischer Informationen in Dokumentensystemen. Der Fokus liegt auf der Integration bestehender Systeme, der Reduktion manueller Suchvorgänge sowie der Einhaltung datenschutzrechtlicher Anforderungen.

{\let\clearpage\relax
\chapter{Stakeholder Interviews}}
\label{sec:stakeholder_interviews}
\section{Vorgehensweise und Zielstellung}
Im Rahmen der Projektplanung wurde eine umfassende Stakeholderanalyse durchgeführt, um die Anforderungen zentraler Interessensgruppen an das geplante Entity-Recognition- und Retrieval-System für medizinische Dokumentensysteme zu identifizieren. Ziel war es, sowohl funktionale als auch nicht-funktionale Anforderungen zu erfassen und dabei die unterschiedlichen Perspektiven aus klinischem Betrieb, technischer Umsetzung und regulatorischem Rahmen zu berücksichtigen.
Die Analyse erfolgte anhand von zwei Interviews mit zentralen Stakeholdergruppen:
\begin{itemize}
	\item Interview 1: mit medizinischem Fachpersonal (Arzt) und Klinikdirektion
	\item Interview 2: mit dem IT-Spezialisten sowie dem Datenschutzbeauftragten der Klinik
\end{itemize}
Die Interviews wurden qualitativ ausgewertet und entlang thematischer Schwerpunkte kategorisiert. Die Ergebnisse flossen direkt in die Anforderungsdefinition und die weitere Projektstrukturierung ein.
\section{Ergebnisse Interview 1: Medizinisches Personal und Klinikdirektion}
\underline{Aktueller Stand und Herausforderungen}
Das medizinische Personal arbeitet täglich mit einer Vielzahl von Dokumenten (z.B. Befunde, Arztbriefe, Laborberichte). Die Informationssuche erfolgt bislang über verschiedene Teilsysteme ohne zentrale Schnittstelle. Dies führt zu ineffizienten Prozessen mit hohem Zeitaufwand (5–20 Klicks pro Abfrage, ca. 100 Zugriffe täglich). Die fragmentierte Systemlandschaft erschwert eine schnelle, kontextbezogene Entscheidungsfindung.
\underline{Anforderungen und Wünsche}
Zentrale Anforderungen sind:
\begin{itemize}
	\item ein kontextsensitives Such- und Dashboardsystem zur automatisierten Darstellung relevanter Informationen
	\item Single Sign-On über alle Subsysteme hinweg
	\item ein modulares, barrierefreies UI mit Unterstützung für Sprachsteuerung
	\item nutzerrollenspezifische Zugriffskonzepte
	\item eine Antwortzeit von unter 3 Sekunden für Standardabfragen
	\item personalisiertes Dashboard mit Speicherung individueller Präferenzen
\end{itemize}
Die Klinikdirektion betont darüber hinaus:
\begin{itemize}
	\item die Effizienzsteigerung als übergeordnetes Ziel zur Entlastung von Personalressourcen
	\item die Einhaltung gesetzlicher Datenschutz- und Interoperabilitätsanforderungen
	\item die Skalierbarkeit des Systems für den klinikweiten Einsatz
	\item die Notwendigkeit frühzeitig testbarer Prototypen und medienwirksamer Präsentationen nach ca. 1,5 Jahren Projektlaufzeit
\end{itemize}
Die Projektumsetzung soll in einem modularen Rollout erfolgen, beginnend mit einem Pilotbetrieb auf einer Station (ca. 100 Patienten).
\section{Ergebnisse Interview 2: IT-Spezialist und Datenschutzbeauftragter}
\underline{Aktueller Stand und technische Rahmenbedingungen}
Die bestehende IT-Landschaft besteht aus über 40 Subsystemen, betrieben in einem eigenen Rechenzentrum mit redundanter Infrastruktur. Externe Cloud-Nutzung ist ausgeschlossen, HL7 und FHIR sind zentrale Standards.
\underline{Datenschutz und Sicherheit}
\begin{itemize}
	\item Strenge DSGVO-konforme Datenverarbeitung (Pseudonymisierung, Verschlüsselung, RBAC, 2FA).
	\item Verarbeitung ausschließlich im geschützten Intranet, keine lokale Speicherung auf Endgeräten.
	\item Nutzung pseudonymisierter Testdaten und restriktiver Zugriff auf Live-Daten.
\end{itemize}
\underline{Systemintegration und Wartung}
\begin{itemize}
	\item Containerisierte Microservice-Architektur (Docker, Kubernetes) und abgestimmte Schnittstellen.
	\item Wartung und Updates durch interne IT, unterstützt durch externe Partner ohne direkten Infrastrukturzugriff.
\end{itemize}
\underline{Leistungsanforderungen und Skalierung}
\begin{itemize}
	\item Unterstützung von 50 Nutzern im Pilotbetrieb, später mehrere Hundert.
	\item Antwortzeiten <3 Sekunden für Standardabfragen, komplexe Analysen in wenigen Minuten.
	\item Einsatz leistungsstarker Hardware mit priorisierten Ressourcen.
\end{itemize}
\underline{Zeitrahmen}
\begin{itemize}
	\item Projektlaufzeit: 3 Jahre
	\item funktionale Prototypen nach 1,5 Jahren
	\item abschließender 3-monatiger Testbetrieb.
\end{itemize}
\section{Bewertung und Nutzen für das Projekt}
Die Stakeholderinterviews ermöglichten ein tiefes Verständnis der bestehenden Schwachstellen, Prioritäten und Erwartungen. Die Kombination aus medizinischer, technischer und datenschutzrechtlicher Perspektive erlaubt eine differenzierte Anforderungsanalyse.
Die Projektplanung profitiert von einem klar definierten Zielbild eines kontextsensitiven, modularen Systems, das eine nutzerspezifische Anpassbarkeit ermöglicht. Die technische und regulatorische Machbarkeit kann frühzeitig validiert werden, beispielsweise durch den Einsatz synthetischer Daten und die Nutzung der bestehenden internen Infrastruktur. Durch den Abgleich der Anforderungen mit den verfügbaren Ressourcen entsteht eine realistische Roadmap für die Umsetzung. Zudem wird die Kommunikation und Einbindung der Stakeholdergruppen verbessert, etwa durch den Einsatz agiler Sprints, klar definierter Meilensteine und regelmäßiger Abstimmungsgremien. Schließlich werden kritische Erfolgsfaktoren wie Benutzerfreundlichkeit, Datenschutzkonformität und das Antwortzeitverhalten frühzeitig identifiziert und in die Planung integriert.


