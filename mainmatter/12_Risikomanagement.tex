\chapter{Risikomanagement}
\label{sec:risikomanagement}

\section{ Risikoplanung}
Ziel der Risikoplanung ist es, potenzielle Gefahrenquellen im Projektverlauf frühzeitig zu erkennen, zu bewerten und geeignete Maßnahmen zur Risikovermeidung oder -Minderung zu definieren. Im Kontext unseres medizinischen Entity-Recognition-Systems sind insbesondere Qualität, Datenschutz, technische Integrität und Projektdurchlaufzeit zentrale Risikofaktoren, da sie direkte Auswirkungen auf Patientensicherheit, Klinikprozesse und regulatorische Vorhaben haben.
\section{Vorgehen}
Die Risikoplanung besteht nach ISO 31000 auf den folgenden Schritten:
\begin{enumerate}
	\item Identifikation potenzieller Risiken
	\item Kategorisierung und Bewertung nach Eintrittswahrscheinlichkeit und Schadenshöhe
	\item Ableitung geeigneter Gegenmaßnahmen
	\item Integration der Maßnahmen in die Projektplanung
	\item Laufende Überprüfung und Anpassung im Projektverlauf
\end{enumerate}
\section{Identifizierte Risiken und Bewertung}
\begin{center}
	\begin{tabular}{|p{4.5cm}|p{2.5cm}|p{2.5cm}|p{2.5cm}|p{3cm}|}
		\hline
		\textbf{Risiko} & \textbf{Eintritts- wahrscheinlichkeit} & \textbf{Schadens- ausmaß} & \textbf{Risikowert} & \textbf{Kategorie} \\
		\hline
		Verzögerung durch schlechte Datenqualität & Mittel & Hoch & Hoch & Datenmanagement \\
		\hline
		Unklare Anforderungen / Scope Creep & Hoch & Mittel & Hoch & Projektsteuerung \\
		\hline
		Ausfall von Schlüsselpersonen (Krankheit, Kündigung) & Mittel & Hoch & Hoch & Personalrisiko \\
		\hline
		Technologische Inkompatibilität (z.B. KI-Modelle <-> KIS-Systeme) & Niedrig & Hoch & Mittel & Technikrisiko \\
		\hline
		Sicherheitslücke / DSGVO-Verstoß bei Zugriffskontrolle & Niedrig bis Mittel & Sehr hoch & Hoch & Recht / Security \\
		\hline
		Unterfinanzierung durch falsch kalkulierte Betriebskosten & Niedrig & Hoch & Mittel & Finanzen \\
		\hline
		Zeitliche Engpässe in Test- und Abnahmephasen & Mittel & Mittel & Mittel & Qualität / Prozess \\
		\hline
	\end{tabular}
\end{center}
\section{Gegenmaßnahmen}
\begin{center}
	\begin{tabular}{|p{7cm}|p{9cm}|}
		\hline
		\textbf{Risiko} & \textbf{Geplante Gegenmaßnahme} \\
		\hline
		Schlechte Datenqualität & Validierung durch ETL-Tools, frühe Tests mit Musterdaten \\
		\hline
		Scope Creep / Anforderungsunsicherheit & Agile Methode mit Backlog-Pflege und Change Requests \\
		\hline
		Ausfall Schlüsselpersonen & Vertretungsregelung, Wissensdokumentation, Pair Programming \\
		\hline
		Inkompatible Technologien & Frühzeitige Prototypen, Schnittstellentests \\
		\hline
		DSGVO-Verstoß / Sicherheitslücke & Security-Audits, Datenschutzschulungen, automatisierte Tests \\
		\hline
		Falsche Kostenannahmen & Monatliche Kostenüberwachung, Kostenpuffer \\
		\hline
		Zeitengpässe in QS & QS kontinuierlich im Sprint, nicht nur zum Projektende \\
		\hline
	\end{tabular}
\end{center}

\newpage
\newpage
